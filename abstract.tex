\begin{abstract}

    Qiskit is an open-source software development kit widely used for learning quantum computation and working on related topics. It is used to run quantum algorithms both on simulation of quantum computers and on real quantum computers. It translates common programming languages like Python into quantum machine language.
    
    \vspace{3mm}
    
    Just like classical computers, quantum computers are also based on gates. They are called Quantum Gates. We can do all sorts of operations with those gates. But the working of those gates is quite different from that of the classical logic gates. 
    
    \vspace{3mm}
    
    Braitenberg Vehicles are very simple devices, yet the resulting behaviour may appear complex or even intelligent. With the aid of Quantum Computers an excellent future application of Quantum Roots can be at our hands. As a practical example, quantum-controlled Braitenberg vehicles is a mobile quantum system and hence acts as a quantum robot.
    
    \vspace{3mm}
    
    I read a paper and tried to simplify the quantum circuits and algorithms. The aforementioned quantum circuit is designed with the help of IBM quantum experience.



    \vspace{5mm}

    \textbf{Keywords:} \textit{Braitenberg Vehicles,
        % Quantum Computation,
        Quantum Circuit,
        Quantum Gates,
        Quantum Robot,
        Qiskit,
        IBM Quantum Experience.
        }

\end{abstract}